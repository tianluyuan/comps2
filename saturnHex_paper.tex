%\documentclass[12pt]{article} % use larger type; default would be 10pt
\documentclass[preprint]{revtex4-1} % APS style

%\usepackage{titling} % Conflicts with aps styling
\usepackage{appendix}
%\usepackage{natbib}
\usepackage{amssymb, amsmath} % for the \begin{align}... \end{align}
                              % command 
\usepackage[lofdepth, lotdepth]{subfig}
\usepackage{graphicx}  % for the \includegraphics command
% \setlength{\droptitle}{-6em}
% \usepackage{array}
% \usepackage{multirow}
% \usepackage{setspace}
% \doublespacing
\usepackage{hyperref}

\begin{document}

\title{Exploration of a Hexagonal Structure on Saturn's Northern Pole}
\author{Tianlu Yuan}
\affiliation{University of Colorado Boulder}

\begin{abstract}
Images from Voyager revealed a hexagonal structure near Saturn's
northern pole at $\approx 77^{\circ}$ north planetographic latitude.
Ground based observations and images from the Hubble Space Telescope (HST) in the early 90s,
along with Cassini data from recent years confirm that the structure still
exists and appears relatively unchanged.  Over the decades spanning its
discovery, several theoretical descriptions of the phenomena have been
offered.  Allison et al. proposed a Rossby wave model sustained by
perturbative forcing from a nearby vortex \cite{Allison1990}.  A more
recent model by Barbosa Aguiar et al., based on barotropic instabilities caused by jets to the north
and south of the hexagon's zonal flow, supported theory with
experimental results by producing polygonal shapes in fluid flows
within the laboratory \cite{BarbosaAguiar2010}.  Numerical simulations
confirming this theory to an extent were also conducted
\cite{MoralesJuberias2011}.  Here, I present an overview of the
observational data collected on the Saturnian north polar hexagon.  I
also discuss the two theoretical models that were put forth to explain the cause of its
six-sided structure and their correctness in light of new
observational data and experimental results.
\end{abstract}
\maketitle

\section{Introduction}
\label{sec:intro}
In the 1980s images of Saturn taken by Voyager led to Godfrey's surprising
discovery of a hexagonal structure on the planet's north pole
\cite{Godfrey1988}.  Now, over three decades later, Saturn's north
polar hexagon remains, superficially unchanged and not entirely
understood.  Voyager's iconic images proved a challenge to explain
theoretically, but with additional data from ground
based observations and the HST in the early
90s, and images from the Cassini mission more recently, our knowledge
of many physical parameters forming the hexagon has increased.  Along
with observation, laboratory experiments and numerical simulations
have helped foster greater understanding of the possible causes
of the jet's six-sided shape.

This paper first details the discovery of the hexagon by Godfrey and
his initial measurements of the structure's velocity field and
rotation rate \cite{Godfrey1988, Godfrey1990}.  A brief discussion of ground based and HST observations
follow that support the earlier velocity measurements
\cite{Caldwell1993, SanchezLavega1993}. I next provide a
theoretical description of the hexagon as  
a stationary Rossby wave, suggesting that its overall structure
is sustained by perturbations from a nearby
vortex observed in the Voyager images \cite{Allison1990}.  Finally, I discuss recent
observational and experimental data and a different theoretical
treatment of the hexagon based on barotropic instability theory
\cite{Fletcher2008, Baines2009, BarbosaAguiar2010, MoralesJuberias2011}.
I also give a basic overview of
some important fluid dynamic concepts in the appendix.  Much of the
technical terminology can be found there.  
I hope that this provides some intuition for understanding 
the theory, but a full, mathematical treatment of advanced concepts will be skipped.

\section{Discovery and Initial Measurements}
\label{sec:discv}
The series of images of Saturn taken by Voyager 1 and 2 were mainly of Saturn's
equatorial region.  In order to visualize the north
pole, Godfrey accounted for the spherical distortion effect by polar projecting the
equatorial images and stitching them together as seen in Figure
\ref{fig:saturnHexs}.  He discovered a hexagonal
structure at $77^{\circ}$ that appeared to be moving at
$6.3 \pm 8 \times 10^{-8}$rad/s relative to the 
Saturnian Radio (SR) rotation period  \cite{Godfrey1988}.  The SR
period is thought to be due 
to Saturn's magnetic field, which in turn reflects conditions within
the conductive interior of the planet.  Hence, it is believed to give
an accurate measure of the rotation rate of Saturn's interior and
indicates that the hexagon is stationary relative to Saturn's interior
rotation within uncertainties \cite{Godfrey1988}.  For the rest of
this paper, all quoted velocites will be relative to the SR rotation.

\begin{figure}
  \centering
  \includegraphics[width=0.9\textwidth]{Figures/morales_3hexagons.jpg}
  \caption{Images of the Saturn polar hexagon as viewed by Voyager
    (left), Cassini-ISS (middle), and Cassini-VIMS (right).  The
    impinging vortex is visible along the bottom left edge of the
    Voyager-taken image \cite[fig~1]{MoralesJuberias2011}.} 
  \label{fig:saturnHexs}
\end{figure}

Along with the hexagon, a large anticyclonic vortex along one of the
edges of the hexagon can be seen in Figure \ref{fig:saturnHexs}, and
its movement was associated with that of the hexagon itself
\cite{Godfrey1988, MoralesJuberias2011}.  In a follow-up paper, Godfrey assumed the
rotation rate of the hexagon to be equivalent to that of the vortex.
By measuring the relative velocity of the vortex, he concluded the
rotation period of the polar hexagon to be $-8.13 \pm 0.52 \times
10^{-9}$rad/s, again suggesting an association between the hexagon
and the planet's interior \cite{Godfrey1990}.

By tracking invidual cloud features within the hexagon, Godfrey
measured the mean zonal velocity of the flow at the center of the hexagon to be
$\approx 100$m/s \cite{Godfrey1988}.  This result is a measure of the
latitudinal velocity averaged over a zonal region via a nearest
neighbor fit routine. Eastward movement was taken to be positive.  The
mean zonal velocity falls off 
moving latitudinally away from the center of the hexagon such that the
the flow drops to $\approx -20$m/s to the south and $\approx 10$m/s to
the north \cite{Godfrey1988}.  A latitudinal profile of the mean zonal
velocity indicates an approximately Gaussian distribution as indicated
in Figure \ref{fig:baines_zonalV}.  This
velocity profile would prove to be important for theoretical
treatments of the hexagon and will be a focus of discussion later
\cite{BarbosaAguiar2010, Allison1990, MoralesJuberias2011}.

Apart from the two Voyager flybys, observations of Saturn with ground
based instruments and the HST served to confirm Godfrey's earlier
analyses.  Using a 1.05m diameter telescope at Pic-du-Midi
Observatory, a group led by Sanchez-Lavega observed Saturn's north
pole over a period from July 1990 to December 1991
\cite{SanchezLavega1993}.  During this time, Saturn held a favorable
orientation relative to Earth that allowed views of the polar hexagon
and the associated vortex.  The vortex's central longitude and latitude were
measured and by combining the Pic-du-Midi data with Voyager's, a mean
rotation rate of $-1.17 \times 10^{-8}$rad/s over the 11 year period was cited \cite{SanchezLavega1993}.  Over a similar time period, the HST
also took several images of Saturn's polar region.  Focusing on
measuring the position of the center of the vortex accurately,
Caldwell's group defined the center of the planet as the center of the
ellipse defined by Saturn's rings \cite{Caldwell1993}.  This was
repeated for each image, and the longitude and latitude of the center
of the polar spot was determined with high accuracy.  A long term
drift rate of $-1.15 \times 10^{-8}$rad/s was reported, in fair
agreement with both Godfrey's and Sanchez-Lavega's measurements
\cite{Caldwell1993}.  It should be noted that all three
measurements cited the uncertainty in the Saturnian rotation
period as a possible systematic in their calculations.  Nevertheless,
the agreement between Voyager, Pic-du-Midi, and the HST data strongly
indicated that the hexagon was both a stationary and long term feature
on Saturn.

\section{Early Hypotheses}
\label{sec:earlyThry}
The Voyager observations led Godfrey to discuss four possible causes
of the stationary rotation rate of the hexagon \cite{Godfrey1988}.
First, stated as being unlikely, is that the rotation rate
could just be a coincidence, since a stationary rotation rate is as good 
as any other.  Another possibility is forcing, from either above or below,
corresponding to Saturn's rotation, that creates the hexagonal wave.
The third possibility regarded the hexagon as an aurora instead of
cloud patterns, and the fourth that the radio rotation rate of Saturn
is actually being generated by the hexagon instead of the planetary
interior \cite{Godfrey1988}.  Aside from the second, these theories
all seem fairly far fetched, which is understandable considering the
amount of data existing at the time.  The third suggestion is easily
refuted by Cassini imaging data \cite{Baines2009}, while the fourth
would require a direct connection between the hexagon and Saturn's
magnetic field.  If we assume that the hexagonal features are clouds,
they would have to be ammonia crystals that could only occur within
the troposphere, not the ionosphere, and its low conductivity would be unlikely to
affect the magnetic field \cite{Gierasch1989}.

The second possibility of forcing can be divided into either forcing
from below or forcing from above.  Godfrey maintains that forcing from
above could be due to aurora features close to the hexagon
\cite{Godfrey1988}. However, this would suggest that there
exists a similar hexagon in the Southern region due to symmetrical
north-south aurora patterns as seen in Figure
\ref{fig:saturnAurora}.   Images from Cassini, Figure \ref{fig:saturnPoles},
again show that this is not the case \cite{Fletcher2008}.  Further, the
atmospheric depth of the hexagon and its seasonal independence
strongly suggests that solar effects are not its cause \cite{Baines2009}.  Forcing
from below seems to be the last realistic hypothesis.  Gierasch
supported this theory by positing thermal convection from the interior as
the cause of the stationary hexagon \cite{Gierasch1989}.  The limited
data on Saturn's interior, however, constrains this interpretation to
opinion, not fact. 

\begin{figure}
  \centering
  \includegraphics[width=0.7\textwidth]{Figures/Saturn-aurora.jpg}
  \caption{Credit J. Trauger (JPL), NASA, \cite{APOD2001}.  Image of
    Saturn's polar UV-aurora taken by the HST.} 
  \label{fig:saturnAurora}
\end{figure}

\section{The Rossby Wave Theory}
\label{sec:rossby}
The first mathematical description of Saturn's hexagon was the Rossby 
wave theory offered by Allison, Godfrey, and Beebe
\cite{Allison1990}. Rossby waves are large scale planetary waves
characterized by low frequency modes and a non-zero Coriolis parameter
\cite{Pedlosky87, PlumbNotes}.  It is typically used in studies
of large-scale, low-frequency waves in Earth's atmosphere, but
Saturn's hexagon also exhibits its properties \cite{Allison1990}.  The large
planetary scale of Saturn's hexagon, as well as its westward drift
relative to the mean background zonal flow are distinguishing features
of Rossby waves \cite{PlumbNotes}.  With regards to its westward phase drift,
recall that the hexagon appears stationary while the jets within move
at high velocities eastward, this is allowed by the Rossby wave
dispersion relation (a measure of the wave velocity), given in Equation
\ref{eq:rossbyDispersion}, which 
indicates that stationary waves can exist if $U>0$ \cite{PlumbNotes}.
Finally, it completes six patterns about a latitudinal region, thus
its wavenumber is six.

Two factors contribute to the
restoring force of Rossby wave oscillations. First, the gradient of the
planetary vorticity as defined by Equation \ref{eq:coriolisf}, $\beta = 2 \Omega
\cos(\theta)/a$, where $\Omega$ 
is the planetary rotation rate, $a$ 
its radius, and $\theta$ the latitude.  Second, the negative
curvature of the mean zonal flow in the meridional (y) direction, $-u_{yy}$
\cite{Allison1990}. This can be thought of as the gradient of the
relative vorticity discussed in the appendix.  The hexagon's zonal
velocity profile can be modeled as a 
Gaussian function of 
the meridional distance from center \cite{Allison1990}.
In the $\beta$-plane approximation, which takes a latitudinal strip
around a sphere and approximates it as a plane, simplifying the
spherical geometry into a planar geometry \cite{PlumbNotes}, it can be
shown that the planetary vorticity gradient is negligible.  First note
that the $1/e$ distance, $L_e$, from the mean in our Gaussian
velocity profile is $\approx 1800$km \cite{Allison1990}.  In the $\beta$-plane
approximation, for latitudes within this $L_e$ region the total
vorticity gradient is dominated by the mean zonal flow
\cite{Allison1990}.  Intuitively, this is because a Gaussian bell curve has
maximal negative curvature near its mean.  The hexagon's velocity
profile is well-modeled by a Gaussian as seen in Figure
\ref{fig:zonalV_vorticityGrad}, and thus the planetary
vorticity gradient can be neglected in the Rossby wave model \cite{Allison1990}.

These approximations allows the calculation of the Rossby phase speed
for barotropic waves (a brief exposition of barotropic fluids can be
found in the appendix) \cite{Allison1990},
\begin{align} 
  \label{eq:rossbyDispersion}
  c = U - \langle -U_{yy} \rangle_e \left(\frac{r}{n}\right)^2.
\end{align}
Here, $c$ is the \textit{horizontal} phase velocity, $n$ the zonal
wave number, and $r$ the radius of the latitudinal circle, and $U$ is
our Gaussian velocity profile \cite{Allison1990}.  For the hexagon,
$r\approx 1.4 \times 10^7$m, $U\approx 100$m/s, $\langle -U_{yy}
\rangle_e \approx 2.2 \times 10^{-11}$m$^{-1}$ s$^{-1}$ and $n=6$,
giving $c \approx 0$ as expected for a stationary wave
\cite{Allison1990}.  However, it is important to note that this
approximation neglects the vertical structure of the wave and its
strict meridional confinement \cite{Allison1990}.  Nevertheless, the
vertical structure can be constrained such that the resulting wave
equation has a solution.
This solution includes a perturbation term, which is assumed to be the
stationary vortex impinging on one side of the hexagon
\cite{Allison1990}.  Furthermore, assuming a
vertically trapped wave with a vertical structure that is
related to the zonal wave number in a specific manner, it can be shown
that the meridional 
confinement of the hexagon is due to this pertubation \cite{Allison1990}.

Although Saturn's vertical structure is not well understood, based on
studies of Earth's 
atmosphere we know that vertically trapped, stationary waves can be
forced from below by 
internal heating \cite{Allison1990}.  This
effectively sets a lower boundary condition that is dependent on thermal variations
in the vertically stratified layers of the atmosphere.  It is
consistent with the possibility of internal forcing as suggested by
Gierasch \cite{Gierasch1989}.  While the Rossby wave model describes
the hexagon well phenomenologically, new data from the Cassini mission
illustrates that the hexagon's structure cannot be due to the
impinging vortex.

\section{Observations by Cassini}
\label{sec:cassini}
\begin{figure}
  \centering
  \includegraphics[width=0.7\textwidth]{Figures/fletcher_saturnPoles.jpg}
  \caption{Saturn's polar temperatures as captured by CIRS.  The
    northern hemisphere is on the 
    right, the southern hemisphere on the left.  \textbf{A} and
    \textbf{B} are at an altitude of 100 mbar in the troposphere,
    \textbf{C} and \textbf{D} at 1 mbar in the stratosphere
    \cite[fig~1]{Fletcher2008}.}
  \label{fig:saturnPoles}
\end{figure}

In 2007 the Cassini probe orbited Saturn at a high latitudinal
inclination and provided the first close-up images of Saturn's poles
since Voyager.  Although the northern pole was shrouded by the
seasonal tilt, the Cassini Composite Infrared Spectrometer (CIRS) took
images of both the north and south poles at mid-infrared
wavelengths \cite{Fletcher2008}.  It thus revealed the polar thermal
distribution seen in Figure \ref{fig:saturnPoles}, and verified the
existence of Saturn's north polar hexagon.  Similar ephemeral
polygonal waves were seen in the south polar region, but none
displayed the permanence of the north polar hexagon
\cite{Fletcher2008}.  Notably, the CIRS images did not contain any resemblance of
the impinging anticyclonic vortex as seen in earlier studies
\cite{Fletcher2008}.

\begin{figure}
  \centering
  \includegraphics[width=0.9\textwidth]{Figures/baines_hexThermalIR.jpg}
  \caption{Polar projection of nine $5.1\mu$m images of Saturn's north
    pole.  Images were obtained in darkness, thus clouds are seen as
    dark in the left images as they block Saturn's $5.1\mu$m thermal
    emission.  The photometrically inverted image on the right shows
    clouds as bright \cite[fig~1]{Baines2009}.} 
  \label{fig:vimsClouds}
\end{figure}

Images taken by the Cassini Visual-Infrared Mapping Spectrometer
(VIMS) at $5.1\mu$m are displayed in Figure \ref{fig:vimsClouds}.
Again, no evidence of the impinging vortex was observed
\cite{Baines2009}.  Zonal wind profile
measurements were made using VIMS images and cloud-tracking methods.
The results are displayed in Figure \ref{fig:baines_zonalV}, and a
maximum mean zonal wind velocity of $124.5 \pm 8.7$m/s was recorded
\cite{Baines2009}.  An analysis conducted assuming constant absolute
vorticity as a function of latitude showed disagreement between
expected and observed mean zonal wind profiles.  In two dimensional
fluids, absolute vorticity tends to be conserved, but in three
dimensions it is the potential vorticity, defined in Equation
\ref{eq:potentialVorticity}, that is conserved under the conditions
given in the appendix \cite{Pedlosky87}.  Cassini
thermal measurements combined with results from the zonal velocity
profile measurements gave a mean potential vorticity value as a
function of latitude.  The results showed that there is a step in
potential vorticity at the hexagon's latitude \cite{Baines2009}.

\begin{figure}[ht]
  \centering
  \includegraphics[width=0.9\textwidth]{Figures/baines_zonalWinds.jpg}
  \caption{Mean zonal velocity
    profile with Cassini data.  The dashed and dotted lines are model
    wind velocities assuming a constant absolute vorticity.  Eastward
    is taken to be positive along the vertical axis \cite[fig~3]{Baines2009}.} 
  \label{fig:baines_zonalV}
\end{figure}

The disappearance of the impinging vortex indicates that the
perturbative Rossby wave theory cannot be correct \cite{Fletcher2008}.
Additionally, the fact that the 
hexagonal structure exists deep into the troposphere and seems
unaffected by seasonal variations rules out solar effects on the
formation of the hexagon \cite{Baines2009}.  Finally, the
disagreement between the hexagon's velocity profile and the velocity
profile expected from a constant absolute vorticity indicates that the third,
vertical dimension plays a crucial role the formation and stability of
the hexagon.  Further, potential vorticity non-conservation indicates
that one or more of the conditions listed in the appendix
does not apply to the hexagon.  This
hints that a new theoretical approach is needed to describe the
hexagon.  Experimental and numerical studies of polygonal flows
arising from barotropic instabilites are one such alternative
\cite{Baines2009}.

\section{Laboratory and Numerical Studies}
\label{sec:expnum}
Rotating polygonal fluid flows have been created in several laboratory
environments.  One study by Jansson et al. demonstrated that stable
polygons can be driven by the rotating bottom plate of a partially
filled cylinder \cite{Jansson2006}.  They speculate that this symmetry
breaking is triggered by the minute wobbling of the bottom plate
\cite{Jansson2006}.  Another study led by Barbosa Aguiar cited
barotropic instabilities in the surrounding zonal jets of the hexagon
to be the driving force behinds its polygonal structure
\cite{BarbosaAguiar2010}.  Barotropic instability can arise from
horizontal shear forces in the flow, which convert
kinetic energy of the zonal flow into kinetic energy of the resulting
eddies \cite{BarbosaAguiar2010, Pedlosky87}. Using two
different experimental setups, 
they generated polygons in a rotating fluid as seen in Figure
\ref{fig:exppolys}, and demonstrated the 
existence of surrounding barotropically unstable regions similar to
those measured around Saturn's hexagon \cite{BarbosaAguiar2010}.  A
third numerical simulation also cites zonal jet instabilities as a
direct link in the production of a stable polygon flow
\cite{MoralesJuberias2011}.  These experimental results are worth
elaborating.  Their analyses leads to the conclusion that Saturn's hexagon
might well arise from similar flow instabilities as the polygons in
Earth's experiments.

\begin{figure}
  \centering
  \subfloat[Pentagon \cite{Jansson2006}]{\label{subfig:jansson}\includegraphics[width=0.3\textwidth]{Figures/jansson_pentagon.jpg}}
  \subfloat[Ring setup \cite{BarbosaAguiar2010}]{\label{subfig:ring}\includegraphics[width=0.3\textwidth]{Figures/barbosaaguiar_ringHex.jpg}}
  \subfloat[Disk setup \cite{BarbosaAguiar2010}]{\label{subfig:disk}\includegraphics[width=0.3\textwidth]{Figures/barbosaaguiar_diskHex.jpg}}
  \caption{Polygons in rotating fluid flow experiments.  In
    \subref{subfig:jansson} the experimental apparatus is a partially
    filled cylinder with a rotating bottom \cite{Jansson2006}.  The
    other two figures are taken from
    \cite[fig~9,10]{BarbosaAguiar2010} and show results from the two
    experimental setups used by Barbosa Aguiar.} 
  \label{fig:exppolys}
\end{figure}

In the experiment by Jansson et al. the constraint parameters were
the rotating velocity of the bottom plate and the height of the
fluid.  Two different fluids were tested, ethyl alcohol and water, and
both exhibited rotating polygon formation \cite{Jansson2006}.  The
rotation rate of the polygons were all found to be much less than the
rotation rate of the bottom plate, and the number of sides of the
formed polygon increased with rotation frequency and decreased with fluid height
\cite{Jansson2006}.  They attributed this spontaneous axial symmetry
breaking to the wobbling of the bottom plate.

\begin{figure}[ht]
  \centering
  \includegraphics[width=0.8\textwidth]{Figures/barbosaaguiar_zonalVelocity.jpg}
  \caption{From \cite[fig~1]{BarbosaAguiar2010}.  Voyager measured zonal velocity
    profile and Saturn's vorticity gradient as a function of
    latitude.  The horizontal dashed line at $\approx 77^{\circ}$ correspond to
    the hexagon.  On the right plot, the dashed line
    correspond to $\beta$, the solid line $u_{yy}$.} 
  \label{fig:zonalV_vorticityGrad}
\end{figure}

A more direct analogue experiment to Saturn's hexagon involved testing
the assumptions of barotropic instability as a possible cause of the
hexagon.  A necessary condition of barotropic instability is the
Rayleigh-Kuo criterion, 
\begin{align} 
  \label{eq:rayleighkuo}
  \beta - u_{yy}<0,
\end{align}
where $\beta = df/dy$ and $u_{yy}$ is the curvature of the zonal
velocity profile along the northward direction
\cite{BarbosaAguiar2010}.  As seen in Figure 
\ref{fig:zonalV_vorticityGrad}, the Rayleigh-Kuo criterion is violated
on either side of the hexagon's latitude at $\approx 77^{\circ}$
\cite{BarbosaAguiar2010}.  However, satisfying the Rayleigh-Kuo
criterion alone is not sufficient for barotropic instability.  If it
were, one might wonder as to why no polygonal shape occurs in the
southern polar region or anywhere else on Saturn for that matter.
After all it seems that the Rayleigh-Kuo criterion is violated in
several regions of latitude below the hexagon.  A theoretical model
based upon a linearised barotropic equation was solved for a measured,
Saturnian zonal velocity profile and several radii of deformation, $L_D$
\cite{BarbosaAguiar2010}.   The Rossby radius of deformation can be
intuitively thought of as the horizontal length scale at which rotational effects
(e.g. Coriolis force) become as important as buoyancy
effects (e.g. gravity) \cite{Gill82}.  The solution to the linearised
barotropic equation gave the wavenumber of maximal growth rate as a
function of $L_D$.  In the case of a wavenumber of 6, the
corresponding $L_D$ was found to be $2500$km, whereas $L_D$ for a jet
at the hexagon's latitude is estimated to be $1135$km \cite{BarbosaAguiar2010,
  MoralesJuberias2011}.  In the case of southern polar
jets, solutions were found to peak at infinite wavenumbers
\cite{BarbosaAguiar2010}. 

\begin{figure}
  \centering
  \includegraphics[width=0.7\textwidth]{Figures/barbosaaguiar_setup.jpg}
  \caption{The setup of the laboratory experiment conducted by Barbosa
  Aguiar et al.  Two differentially-rotating sections were employed,
  a ring and a disk, to force out a zonal jet profile
  \cite[fig~4]{BarbosaAguiar2010}.}  
  \label{fig:labSetup}
\end{figure}

In order to investigate this topic further, a laboratory model was
developed using the two slightly differing setups shown in Figure
\ref{fig:labSetup}. 
Both involved cylinders 
rotating at $\Omega$, but the distinguishing
feature is the separate differentially-rotating section
\cite{BarbosaAguiar2010}.  This section could either be a disk in
contact with both the top and bottom of the fluid or a
ring in contact with only the upper fluid surface, and its
differential rotation forces out a jet-like flow from 
the body fluid \cite{BarbosaAguiar2010}.  A non-zero $\beta$ parameter
could be simulated by conical bottoms sloping away from the center.  The
results for both setups  
are shown in Figure \ref{fig:exppolys}, and their measured mean zonal
flow velocities and vorticity gradients are shown in Figure
\ref{fig:labFlowVorticity}.  These appear similar to the profiles
observed for Saturn's hexagon, and violation of the Rayleigh-Kuo
criterion occurs on either side of the jet flow \cite{BarbosaAguiar2010}.

\begin{figure}
  \centering
  \includegraphics[width=0.8\textwidth]{Figures/barbosaaguiar_expResults.jpg}
  \caption{The mean zonal velocity (left) and vorticity gradient
    (right) as measured from analysis of flow images in laboratory
    experiments using the ring setup.  On the right, $\beta=0$
    corresponds to the dashed line, and violation of the Rayleigh-Kuo
    criterion  is evident.  Also, note how the profiles appear
    similar to the observed Saturnian hexagon's profile in Figure
    \ref{fig:zonalV_vorticityGrad} \cite[fig~6]{BarbosaAguiar2010}.} 
  \label{fig:labFlowVorticity}
\end{figure}

Additionally, a numerical simulation based on the Explicit Planetary
Isentropic-Coordinate (EPIC) model demonstrated the possible formation
of stable polygons as a result of instabilities arising from
zonal jet nonlinear equilibrations \cite{MoralesJuberias2011}.  They
found that the zonal wave number depended strongly on $u_{yy}$, as
speculated in previous studies.  By initially seeding the simulation with
a Gaussian velocity profile, polygons formed and propagated with
velocities given by the Rossby wave dispersion relation, as suggested
years earlier by Allison 
et al. \cite{Allison1990, MoralesJuberias2011}.  However, their
simulated propagation rate did not match the hexagon's observed
rotation.  A modified 
initial velocity profile was developed, with an additional term that
``slowed down'' the wave propagation \cite{MoralesJuberias2011}.  The
simulated jets violated the Rayleigh-Kuo criterion, and the dominant
instability mode was speculated to be barotropic instabilities in
agreement with the Barbosa Aguiar experiment discussed above
\cite{MoralesJuberias2011}.

\section{Conclusion}
\label{sec:conclusion}
Although the exact cause behind Saturn's north polar hexagon
still remains somewhat of a mystery, much progress has been made since
its initial discovery to allow a better understanding of a startling
phenomena.  Starting with Voyager and Godfrey's initial discovery, the
hexagon first appeared alongside an impinging vortex, with a near zero
rotation rate relative to Saturn's radio rotation period
\cite{Godfrey1988}.  Images from
the HST and Pic-du-Midi observatory in the early 90s confirmed both
facts over ten years later \cite{Caldwell1993, SanchezLavega1993}.  The images 
from Voyager bedazzled and baffled scientists, and many early
hypotheses as to the cause of the stationary wave were
speculated.  The most realistic of them, forcing from below due to internal
convective heating, was propounded by both Godfrey and Gierasch
\cite{Godfrey1988, Gierasch1989}, but it wasn't until the Rossby wave
theory that a mathematical description of the nature of the hexagon
was given \cite{Allison1990}.  With the Cassini mission entering into
orbit in 2004, interest in Saturn's hexagon was revived.  It was
discovered that the impinging vortex no longer existed, thus refuting
the perturbative Rossby wave theory given by Allison et
al. \cite{Baines2009}.  Laboratory experiments demonstrated the
possibility of spontaneous polygonal formation in rotating fluids,
with the cause attributed to barotropic instabilities arising in the
zonal jet flows surrounding the polygons \cite{BarbosaAguiar2010,
  Jansson2006, MoralesJuberias2011}.  The laboratory analogue were
found to have zonal flows and vorticity gradients comparable to those observed on
Saturn, and is probably the best explanation of Saturn's
hexagon we have today.

\appendix*
\section{Fluid Dynamics in Brief}
\label{app:fluidDyn}
The following is a summary of some chapters from
Pedlosky's \textit{Geophysical Fluid Dynamics} \cite{Pedlosky87},
Batchelor's \textit{An Introduction to Fluid Mechanics}
\cite{Batchelor67}, and Gill's \textit{Atmosphere-Ocean Dynamics} \cite{Gill82}.  

A fluid flow is, at the most fundamental level, described by a
\textit{velocity field}, \textbf{u}.  In the Eulerian methodology, this requires a velocity
vector to be assigned at every point in the space of the fluid
\cite{Batchelor67}.  It is then of interest to attack the problem of
the time evolution the velocity field,
\begin{align} 
  \frac{d\textbf{u}}{dt} &=
  \frac{\partial\textbf{u}}{\partial t}+\textbf{u}\cdot(\nabla\textbf{u}).
\end{align}
The continuity equation, along with conservation laws, gives an
equation of motion for \textbf{u} in a non-rotating frame
\cite{Pedlosky87}. In a rotating frame, the equation of motion is
modified by the Coriolis acceleration to be,
\begin{align}
  \label{eq:eqm} 
  \rho \left[\frac{d\textbf{u}}{dt} + 2 \boldsymbol{\Omega}\times
    \textbf{u}\right]&= -\nabla p+ \rho\nabla\Phi+\mathcal{F},
\end{align}
where the density $\rho$ has been assumed constant, $\boldsymbol{\Omega}$
is the planetary rotation, $p$ the pressure, $\Phi$ is the
modified potential due to gravity and centripedal
acceleration, and $\mathcal{F}$ is any additional frictional force
\cite{Pedlosky87}.  From Equation \ref{eq:eqm}, we see that the
Coriolis acceleration contributes a factor of $2\boldsymbol{\Omega}\times
\textbf{u}$, estimated as $O(2\Omega U)$.  Taking the ratio of the
relative acceleration $\frac{d\textbf{u}}{dt} \approx O(U^2/L)$ to the
Coriolis acceleration gives the
important \textit{Rossby number} $\epsilon=\frac{U}{2\Omega L}$
\cite{Batchelor67, Pedlosky87}.  The Rossby number is an estimate of
the relative importance of the Coriolis force on the velocity field.
For small Rossby number, the Coriolis acceleration is important.  This
is the case for large scale flows such as Saturn's hexagon as the
relative acceleration becomes small in comparison to the Coriolis acceleration.

% An important approximation to equation \ref{eq:eqm} is the \textit{geostrophic
% approximation}.  If the Rossby number is small and the ratio of
% frictional acceleration to the Coriolis acceleration is small,
% Equation \ref{eq:eqm} can be simplified to first approximation as
% \begin{align}
%   \label{eq:simpleeqm}
%   \rho 2\boldsymbol{\Omega} \times \textbf{u} = -\nabla p + \rho
%   \nabla \Phi
% \end{align}
% \cite{Pedlosky87}.  With a few further order of magnitude assumptions,
% the geostrophic approximation leads to a remarkable fact: the
% horizontal velocity (in spherical coordinates this would be the
% non-radial components of $\textbf{u}$) runs parallel to the lines of
% constant pressure.  This is in direct contrast to the fluid flow in a
% non-rotating system, and arises as a direct result of the balancing
% interplay of $\nabla p$ and the Coriolis acceleration
% \cite{Pedlosky87}.  It is the reason that atmospheric flows are
% represented by isobars on a map.
Given a velocity field, it is useful to define the \textit{vorticity} as
$\boldsymbol{\omega} = \nabla \times \textbf{u}$ and work with
equations of motion in terms of the vorticity instead
\cite{Batchelor67, Pedlosky87}.  Note that there is a direct analogy
between the relationship of vorticity and velocity field to that of
the current density and magnetic field.  However, there is no causal
relationship in the case of vorticity and velocity field; the
vorticity is simply a useful and intuitive concept for understanding
fluid flows.  The planetary vorticity is $2\boldsymbol{\Omega}$ and
the component of the planetary vorticity normal to the planet's
surface is called the Coriolis parameter \cite{Pedlosky87},
\begin{align}
  \label{eq:coriolisf}
  f = 2\Omega \sin \theta.
\end{align}
By treating a thin latitudinal strip on a sphere
as a geometrically flat plane, $f$ can be linearly approximated as $f
\approx f_0+\beta_0y$, where $f_0 = 2\Omega \sin \theta_0$ and $\beta_0 =
\frac{2\Omega}{r_0} \cos \theta_0$ \cite{Pedlosky87}.  This is called
the $\beta$-plane 
approximation, and is used extensively in atmospheric models,
including that of Saturn's hexagon.

In a rotating reference frame, the absolute vorticity is defined as
$\boldsymbol{\omega_a} = \boldsymbol{\omega} + 2\boldsymbol{\Omega}$,
the sum of relative and planetary vorticities.  Starting with Equation
\ref{eq:eqm}, it is possible to derive the vorticity equation \cite{Pedlosky87}:
\begin{align}
  \label{eq:vorticityeq}
  \frac{d\boldsymbol{\omega}}{dt} = \boldsymbol{\omega_a} \cdot \nabla
  \textbf{u}- \boldsymbol{\omega_a} \nabla \cdot \textbf{u} +
  \frac{\nabla \rho \times \nabla p}{\rho^2} + \nabla \times \frac{\mathcal{F}}{\rho}.
\end{align}
The third term on the right hand side is called the baroclinic
vector.  A fluid is baroclinic if $\frac{\nabla \rho \times \nabla
  p}{\rho^2} \neq 0$, and barotropic otherwise.  A barotropic fluid
thus by definition must have coinciding surfaces of constant $\rho$
and $p$, and a relation $\rho = \rho(p)$ can be found
\cite{Pedlosky87}.  Instabilities that arise in both barotropic and
baroclinic fluids play a crucial role in large scale flow dynamics and
the wave patterns that arise in planetary atmospheres.  Several
theoretical descriptions of the Saturnian north polar hexagon use
instability theory as a starting point in modeling its dynamical
behavior.

Finally, one can define the potential vorticity as
\begin{align}
  \label{eq:potentialVorticity}
  \Pi = \frac{\boldsymbol{\omega_a}}{\rho}\cdot \nabla \lambda,
\end{align}
where $\lambda$ is some property such as density or the potential
temperature \cite{Pedlosky87}.  The potential vorticity is a useful
concept as it allows vertical structure to be incorporated via the
parameter $\lambda$ for three dimensional fluids.  It is 
conserved under the following conditions: 
\begin{itemize}
  \item the fluid is barotropic or $\lambda$ is dependent on only
    $\rho$ and $p$,
  \item $\lambda$ is conserved for each fluid element,
  \item the frictional force is negligible \cite{Pedlosky87}.
\end{itemize}

\bibliographystyle{plain}	% (uses file "plain.bst")
\bibliography{refs}
\end{document}

