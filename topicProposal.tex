
\documentclass[11pt]{article} % use larger type; default would be 10pt

\usepackage{titling}
\usepackage{natbib}
\usepackage{amssymb, amsmath} % for the \begin{align}... \end{align}
                              % command 
\usepackage{graphicx}  % for the \includegraphics command
% \setlength{\droptitle}{-6em}
% \usepackage{array}
\usepackage{multirow}
\usepackage{booktabs} % for much better looking tables

% Subtitle script taken from
% http://tex.stackexchange.com/questions/50182/subtitle-with-the-maketitle-page
\newcommand{\subtitle}[1]{%
  \posttitle{%
    \par\end{center}
    \begin{center}\large#1\end{center}
    \vskip0.5em}%
}

\begin{document}

\title{Exploration of a Hexagonal Structure on
  Saturn's Northern Pole}
\subtitle{Topic Proposal}
\author{Tianlu Yuan}
\maketitle

In the 1980s images of Saturn taken by Voyager led to Godfrey's surprising
discovery of a hexagonal structure on the planet's north pole
\cite{Godfrey1988}.  The absolute rotation period of the hexagon was derived by
Godfrey to be 10 hours 39 min $22.082\pm0.122$ s \cite{Godfrey1990},
corresponding to a velocity of  $+109 \pm 6.9 \times 10^{-3}$m/s
relative to the radio rotation period of the planet \cite{Godfrey1990}.
This slow relative velocity of the hexagon led to the theory that the
structure could be interpreted as a stationary Rossby wave as
suggested by Allison, Godfrey, and Beebe \cite{Allison19900203}.

With more recent data from the Cassini mission we know that the
hexagonal structure still exists and appears relatively unchanged
\cite{BarbosaAguiar2010}.  This, along with images taken by the Hubble
Space Telescope, have fostered interest in the scientific community
regarding the hexagon's formation.  Laboratory
experiments on Earth have demonstrated that polygons can form on a
rotating fluid in such a way that axial symmetry is broken
``spontaneously'' \cite{Jansson2006}.  More complex models based on
laboratory studies of the instability of vertical jets and shear
layers and analysis of the saturnian zonal wind profile have led to the theory that Saturn's
hexagonal structure is the result of a nonlinear equilibration 
of unstable zonal jets\cite{BarbosaAguiar2010}.  

In summary, our understanding of Saturn's hexagonal structure is still
incomplete.  Numerous attempts have been undertaken to explain this
phenomena through some combination of observation of Saturn and
experimentation on Earth.  Understanding this fascinating phenomena
will surely spur new research in geophysics, planetary physics,
and fluid dynamics in the near future.

\bibliographystyle{plain}	% (uses file "plain.bst")
\bibliography{refs}
\end{document}

